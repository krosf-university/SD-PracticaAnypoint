\documentclass[12pt]{article}
\usepackage[spanish]{babel}
\usepackage[square,sort,comma,numbers]{natbib}
\usepackage{url}
\usepackage{hyperref}
\usepackage[utf8]{inputenc}
\usepackage{lmodern}
\usepackage{textcomp}
\usepackage{amsmath}
\usepackage{graphicx}
\graphicspath{{images/}}
\usepackage{parskip}
\usepackage{fancyhdr}
\usepackage{vmargin}
\usepackage{listings,xcolor}
\lstset{
    string=[s]{"}{"},
    stringstyle=\color{blue},
    comment=[l]{:},
    commentstyle=\color{black},
}
\hypersetup{
    colorlinks=true,
    urlcolor=blue,
}
\setmarginsrb{3 cm}{2.5 cm}{1 cm}{2.5 cm}{1 cm}{1.5 cm}{1 cm}{1.5 cm}

\title{Práctica: Anypoint}								
% Title
\author{Kevin López Cala \break Carlos Rodrigo Sanabria Flores}					
% Author
\date{\today}													
% Date

\makeatletter
\let\thetitle\@title
\let\theauthor\@author
\let\thedate\@date
\makeatother

\pagestyle{fancy}
\fancyhf{}
\rhead{Sistemas Distribuidos}
\lhead{\thetitle}
\cfoot{\thepage}

\begin{document}

%%%%%%%%%%%%%%%%%%%%%%%%%%%%%%%%%%%%%%%%%%%%%%%%%%%%%%%%%%%%%%%%%%%%%%%%%%%%%%%%%%%%%%%%%

\begin{titlepage}
	\centering
    \vspace*{0.5 cm}
   \textsc{\Large Grado en Ingeniería Informática}\\[0.5 cm]		 % Course Code
    \textsc{\LARGE Sistemas Distrbuidos}\\[0.5 cm]	% University Name
	\rule{\linewidth}{0.2 mm} \\[0.4 cm]
	{ \huge \bfseries \thetitle}\\
	\rule{\linewidth}{0.2 mm} \\[1.5 cm]

	\begin{minipage}{0.4\textwidth}
		\begin{flushleft} \large
			\emph{Autores:}\\
			\theauthor
			\end{flushleft}
			\end{minipage}~
			\begin{minipage}{0.453\textwidth}
			\begin{flushright} \large
            	\emph{Fecha:}\\
				\thedate
		\end{flushright}
	\end{minipage}\\[2 cm]
    \vfill
\end{titlepage}
\pagebreak
\section{Descripción del flujo \textit{subscribe}}
    \begin{itemize}
        \item El nombre del flujo se denomina subscribe.
        \item El sensor se encuentra en toda España.
        \item URL POSIBLE SI NO BORRAR
        \item El período de actualización es de 1400 minutos.
        \item Formato del flujo 
    \begin{lstlisting}         
    "subscribe": {
    "campus": String,
    "chatid": Int,
    "id": String,
    "username": String
  },
    "subscribe_responde": {
    "campus": String,
    "chatid": Int,
    "new": Boolean,
    "username": String
  }
        \end{lstlisting}
    \item Esquema del funcionamiento con descripcion
    \item Capturas de pantalla del funcionamiento
\end{itemize}
\section{Descrición del flujo \textit{image}}
    \begin{itemize}
        \item El nombre del flujo se denomina image.
        \item El sensor se encuentra en toda España.
        \item URL POSIBLE SI NO BORRAR
        \item NO LO SE min.
        \item Formato del flujo
    \begin{lstlisting}         
    "subscribe": {
    "campus": String,
    "chatid": Int,
    "id": String,
    "username": String
  },
    "subscribe_responde": {
    "campus": String,
    "chatid": Int,
    "new": Boolean,
    "username": String
  }
        \end{lstlisting}
    \item Esquema del funcionamiento con descripcion
    \item Capturas de pantalla del funcionamiento
\end{itemize}

\section{Errores}
\begin{itemize}
    \item  Al intentar comparar que un \textbf{\textit{payload}} 
     es nulo tuvimos que poner como condición ,
    \textbf{payload is org.mule.transport.NullPayload}
\end{itemize} 

\newpage
\begin{thebibliography}{X}
\bibitem{Bot Twitter} 
\textsc{Bot Twitter}
\href{http://docs.tweepy.org/en/v3.4.0/getting_started.html}{\textit{Tweepy.}}
\href{https://developer.twitter.com/en.html}{\textit{API Twitter.}}
\bibitem{Bot Telegram} 
\textsc{Bot Telegram}
\href{https://telegraf.js.org/#/}{\textit{Telegraf.}}

\end{thebibliography}

\end{document}